\section{Conclusion}
\label{cha:conclusion}

The Android development world is growing rapidly. It is hard for developers to stay up-to-date all the time. Libraries and APIs are constantly changing. Google tries to counter this problem with Android Jetpack. They provide a collection of libraries to help developers cope with this problem. With Jetpack, developers have all the tools to create a modern android application.

On the other hand, Jetpack also has a not-so-good side. For example, it is hard for people who learned Android programming the older way, like the XML-style UI, to switch to the declarative UI of Jetpack Compose. Everything they learned about fragments is different now, because fragments just don't exist anymore. There are only Composable functions left to interact with the UI.

Jetpack has a great documentation about all components and also the GitHub account of Google is a great starting point. They provide coding examples to get in touch with the best coding practices in Android development. It just takes some time to get used to a new style of programming. But it is definitely recommended for an Android beginner to start with the old XML-style UI because it necessary to learn about the components a application really has and how these work together. Jetpack Compose does a lot of magic in the background and when someone really wants to learn Android development it is necessary to know what these magically short code snippets really do. Currently another big drawback is, that some features of Jetpack are not yet mature enough (for example the preview of Jetpack Compose).

To sum it up, Android Jetpack is really great. Room, Navigation and Android KTX (Kotlin extensions) make most of the developers happy, it simplifies so many things and saves the lifes from bad practices. Jetpack Compose will be the future of UI development, but it will take some time to adapt to this new style. Everybody is looking forward for the new handy libraries in Android Jetpack.