\section{Introduction}
\label{cha:introduction}

Technologies are constantly changing, and this is true for Android as well. The number of libraries for Android developers is increasing rapidly, and therefore it is very difficult to choose the right ones. Furthermore, it is not an easy task to keep the architecture of an app in a manner that is easily understandable for other people. Uncontrollable growth must be prevented.

In 2018, Google has decided to give Android a new coat of paint with Android Jetpack. Android Jetpack should be a suite of libraries which should reflect the best practices in developing an Android application.

Android Jetpack is constructed like LEGO building blocks -- the libraries can be plugged together according to requirements. Splitting a framework into multiple blocks is necessary when it grows into an unmanageable state.

The most important libraries of Android Jetpack are Jetpack Compose, Room, Navigation and Databinding. According to their website \cite{android_jetpack}, Jetpack consists of almost 100 libraries (as of June 2022).

% No Sections!
% - Motivation
% - Challenges\\(Prevent impact of new features on existing concepts)
% - Goals
%    - Division into blocks for growing frameworks
%    - Module/Package (Lego)
%    - Prevent uncontrollable growth
%    - Software in continuous change (new aspects)