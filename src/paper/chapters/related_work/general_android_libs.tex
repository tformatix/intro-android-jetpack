\subsection{General Android Libraries}
\label{cha:general_android_libs}

\noindent
\textbf{Native Libraries}

In addition to the Linux kernel layer, the Android platform contains a number of native libraries. Most of the functionality provided through the Android runtime layer is available through these native libraries \cite{cinar_2015_android_quick_apis_reference}.

According to Cinar \cite{cinar_2015_android_quick_apis_reference}, the most notable of them are:

\begin{itemize}
    \item \textit{SQLite}: in-memory, relational SQL database for persisting and accessing the application's data
    \item \textit{WebKit}: enables HTML, CSS and JavaScript to include web technology
    \item \textit{OpenGL ES}: rendering functionality (2D and 3D)
    \item \textit{Open Core}: record and play back audio and video content
    \item \textit{OpenSSL}: secure communication with SSL/TLS
\end{itemize}

\noindent
\textbf{Android Support Library}

While targeting a lower API level will increase the app's audience, it also limits the Android platform features to use. To overcome this trade-off, the Android Support Library was introduced. The Android Support Library Package is a collection of code libraries that provide backward compatible versions of recent Android APIs.
An important fact is that the Android Support Library does not cover every new API because sometimes new OS features are required \cite{cinar_2015_android_quick_apis_reference}.